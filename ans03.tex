\section{Условные вероятности. Формула полной вероятности. Формула Байеса. Примеры}

\begin{definition}[Условная вероятность]
	Пусть \PSP~---вероятностное пространство, $A, B \in \Flg$. Тогда {\it условной вероятностью} события $A$ при условии, что произошло событие $B$, называют величину
	$$\PP(A|B)=\frac{\PP(A\cap B)}{\PP(B)}, \mbox{если $\PP(B) > 0$}$$
	$$\PP(A|B) = 0, \mbox{если $\PP(B) = 0$}$$
\end{definition}

\begin{statement}
Пусть \PSP~--- вероятностное пространство. Пусть $B\in\Flg,~\PP(B) > 0$. Тогда $\PP(\cdot|B): \Flg \rightarrow \RR$ является счётно-аддитивной вероятностной мерой над измеримым пространством $\langle \oO, \Flg \rangle$.
\end{statement}

\begin{proof}
	Проверим, что для $\PP(\cdot|B)$ выполняется определение вероятностной меры. Действительно:
	$$\forall A \in \Flg~~\PP(A|B)=\frac{\PP(A\cap B)}{\PP(B)}\ge 0$$
	$$\PP(\oO|B)=\frac{\PP(\oO\cap B)}{\PP(B)} = 1$$
	Пусть $\{A_i\}$ --- последовательность попарно непересекающихся событий. Поскольку $\Flg$ --- \gmалгебра, то их объединение $A$ тоже принадлежит $\Flg$. Проверим, что
	$$\PP(A|B)=\sum_{i=1}^\infty\PP(A_i|B)$$
	Действительно,
	\begin{multline}
		\sum_{i=1}^\infty\PP(A_i|B)
		 =\sum_{i=1}^\infty\frac{\PP(A_i \cap B)}{\PP(B)} = \\
		 =\frac{1}{\PP(B)}\sum_{i=1}^\infty\PP(A_i \cap B)
		 =\frac{1}{\PP(B)}\PP(A\cap B) = \\
		 =\PP(A|B)
	\end{multline}
\end{proof}

\begin{theorem}[Формула полной вероятности]
Пусть \PSP~--- вероятностное пространство. Пусть $\oO=\bigsqcup_{i=1}^\infty B_i$, пусть $A, B_1, B_2, \ldots \in \Flg$. Тогда
$$\PP(A)=\sum_{i=1}^{\infty}\PP(A|B_i)\PP(B_i)$$
\end{theorem}

\begin{proof}
Заметим, что $\PP(A\cap B_i) = \PP(A|B_i)\PP(B_i)$ является верным равенством и в случае, когда $\PP(B_i) = 0$, и в случае, когда $\PP(B_i) > 0$. А значит
$$\sum_{i=1}^{\infty}\PP(A|B_i)\PP(B_i)=
\sum_{i=1}^{\infty}\PP(A\cap B_i)=
\PP(A)$$
\end{proof}
Заметим, что в конечных случаях формула полной вероятности тоже работает.

\begin{theorem}[Формула Байеса]
Пусть \PSP~--- вероятностное пространство. Пусть $\oO=\bigsqcup_{i=1}^\infty B_i$, пусть $A, B_1, B_2, \ldots \in \Flg$. Пусть также $\PP(A) > 0$ Тогда
$$\PP(B_k|A)=\frac{\PP(A|B_k)\PP(B_k)}{\sum_{i=1}^\infty\PP(A|B_i)\PP(B_i)}$$
\end{theorem}

\begin{proof}
$$\PP(A\cap B_k) = \PP(A|B_k) \PP(B_k)$$
$$\PP(A\cap B_k) = \PP(B_k|A) \PP(A)$$
$$\PP(B_k|A) \PP(A) = \PP(A|B_k) \PP(B_k)$$
$$\PP(B_k|A) = \frac{\PP(A|B_k) \PP(B_k)}{\PP(A)}$$
осталось расписать $\PP(A)$ по формуле полной вероятности и получить желаемое.
\end{proof}

Смысл формулы Байеса можно понимать так: $B_i$ --- это гипотезы, а $A$ --- результат эксперимента. Нам известна {\it априорная} вероятность $\PP(A|B_i)$ получения результата $A$ при выполнении гипотезы $B_i$. Теперь, зная результат эксперимента, мы хотим узнать {\it апостериорную} вероятность того, что гипотеза $B_k$ верна.

Например, с помощью формулы Байеса можно решать какую-нибудь задачу про смерть лорда Вайла, которого хотят отравить или зарезать дворецкий, сын и жена.

Приведём ещё пример решения задачи про шары помощью формулы Байеса.
\begin{example}
	Пусть в ящике $n$ шаров, из них $k$ белых. Шары извлекаются без возвращений равновероятно. Какова вероятность на $j$-ом шаге вытащить белый шар?

	Итак, вероятностное пространство, соотвествующее $j$-кратному выбору без возвращения таково:

		\begin{itemize}
		\item $\oO = \{a_1a_2 \ldots a_j |~a_i = 1, 2, \ldots, n\}$
		\item $\Flg = 2^\oO$
		\item $P(\{a_1a_2 \ldots a_j\}) = \frac{1}{\frac{(n)!}{(n-j)!}}$
		\end{itemize}

	Доказывать будет индукцией по числу шаров и шагов. Обозначим за $A_{j,n,k}$ событие «вытащить белый шар на $j$-ом шаге, если в урне изначально было $n$ шаров, из которых $k$ белых» и за $B_{j,n,k}$ событие «вытащить чёрный шар на $j$-ом шаге, если в урне изначально было $n$ шаров, из которых $k$ белых».

	{\it База.} $A_{1,n,k}=\frac{k}{n}$.
	
	{\it Доказательство перехода.} 
	\begin{multline}
	\PP(A_{j,n,k}) = \\
	 = \PP(A_{j,n,k}|A_{1,n,k})\PP(A_{1,n,k}) + \PP(A_{j,n,k}|B_{1,n,k})\PP(B_{1,n,k}) = \\ 
	 = \PP(A_{j-1,n-1,k-1})\PP(A_{1,n,k}) + \PP(A_{j-1,n-1,k})\PP(B_{1,n,k})
	 \end{multline}
\end{example}