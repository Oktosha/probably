\section{Дискретные вероятностные пространства. Классическое определение вероятности. Примеры. Геометрические вероятности. Примеры.}

\subsection{Дискретные вероятностные пространства}

\paragraph{Определения}

\begin{definition}[Дискретное вероятностное пространство]
Вероятностное пространство \PSP~называется {\it дискретным вероятностным пространством}, если \oO~не более чем счётно.
\end{definition}

\begin{definition}[Классическое определение вероятности]
Вероятностное пространство \PSP~называется {\it классическим вероятностным пространством}, если:
	\begin{itemize}
	\item \oO~конечно, $|\oO| = n $
	\item $\Flg = 2^\oO$
	\item $\forall \omega \in \oO~~\PP(\omega) = \frac{1}{n}$
	\end{itemize}
\end{definition}


\paragraph{Примеры классических вероятностных пространств}

\begin{example}[Бросок кубика]
Бросок идеального игрального кубика принято описывать 
вероятностным пространством \PSP~следующего вида:
\begin{itemize}
		\item $\oO = \{1, 2, 3, 4, 5, 6\}$
		\item $\Flg = 2^{\oO}$
		\item $\PP(\{\omega\}) = \frac{1}{6}$, где $\omega \in \oO$
	\end{itemize}
\end{example}

\begin{example}[Равновероятный выбор из $n$ объектов]
В случае равновероятного выбора из $n$ объектов соотвествующее вероятностное пространство \PSP~имеет вид:
\begin{itemize}
		\item $\oO = \{1, \ldots, n\}$
		\item $\Flg = 2^{\oO}$
		\item $\PP(\{\omega\}) = \frac{1}{n}$, где $\omega \in \oO$
	\end{itemize}
\end{example}

\begin{example}[Упорядоченный $k$-кратный выбор из $n$ объектов с возвращением]
Упорядоченный $k$-кратный выбор из $n$ объектов с возвращением описывается вероятностным пространством \PSP~следующего вида:
\begin{itemize}
		\item $\oO = \{a_1a_2\ldots a_k |~a_i = 1, \ldots, n\}$
		\item $\Flg = 2^{\oO}$
		\item $\PP(\{\omega\}) = \frac{1}{n^k}$, где $\omega \in \oO$
	\end{itemize}
\end{example}

\begin{example}[Упорядоченный $k$-кратный выбор из $n$ объектов без возвращения]
Упорядоченный $k$-кратный выбор из $n$ объектов без возвращения описывается вероятностным пространством \PSP~следующего вида:
\begin{itemize}
		\item $\oO = \{a_1a_2\ldots a_k |~a_i = 1, \ldots, n,~~i \not = j \Rightarrow a_i \not = a_j \}$
		\item $\Flg = 2^{\oO}$
		\item $\PP(\{\omega\}) = \frac{1}{\frac{n!}{(n-k)!}}$, где $\omega \in \oO$
	\end{itemize}
\end{example}

\begin{example}[Неупорядоченный $k$-кратный выбор из $n$ объектов с возвращением]
Неупорядоченный $k$-кратный выбор из $n$ объектов с возвращением описывается вероятностным пространством \PSP~следующего вида:
\begin{itemize}
		\item $\oO = \{a_1a_2\ldots a_k |~a_i = 1, \ldots, n~~i < j \Rightarrow a_i \le a_j \}$
		\item $\Flg = 2^{\oO}$
		\item $\PP(\{\omega\}) = \frac{1}{C_{n + k - 1}^k}$, где $\omega \in \oO$
	\end{itemize}
\end{example}

\begin{example}[Неупорядоченный $k$-кратный выбор из $n$ объектов без возвращения]
Неупорядоченный $k$-кратный выбор из $n$ объектов без возвращения описывается вероятностным пространством \PSP~следующего вида:
\begin{itemize}
		\item $\oO = \{a_1a_2\ldots a_k |~a_i = 1, \ldots, n,~~i < j \Rightarrow a_i < a_j \}$
		\item $\Flg = 2^{\oO}$
		\item $\PP(\{\omega\}) = \frac{1}{C_n^k}$, где $\omega \in \oO$
	\end{itemize}
\end{example}

\paragraph{Примеры конечных (неклассических) дискретных вероятностных пространств}

\begin{example}[Распределение Бернулли]
	Вероятностное пространство \PSP~ следующего вида
	\begin{itemize}
		\item $\oO = \{0, 1\}$
		\item $\Flg = 2^{\oO}$
		\item $\PP(\{1\}) = p, \PP(\{0\}) = q$, где $q = 1 - p$
	\end{itemize}
	описывает некоторый однократный эксперимент, в котором $\{1\}$ соответствует успеху, $p$ --- вероятности успеха, а $\{0\}$ и $q$ --- провалу и его вероятности. Распределение вероятностей $\PP: \{\varnothing, \{0\}, \{1\}, \{0, 1\}\} \rightarrow \{0, q, p, 1\}$ называют {\it распределением Бернулли}.
\end{example}

\begin{example}[Схема Бернулли]
	Опыт, состоящий в $n$-кратном повторении некоторого эксперимента с вероятностью успеха $p$, и соответствующее ему вероятностное пространство
	\begin{itemize}
		\item $\oO = \{a_1a_2\ldots a_n~|~~a_i = 0, 1\}$
		\item $\Flg = 2^{\oO}$
		\item $\PP(\{\omega\}) = p^k q^{n-k}$, где $q = 1 - p$, $k = |\omega|_1$
	\end{itemize}
	называют {\it схемой Бернулли}
\end{example}

\begin{example}[Испытание с разновероятными исходами]
	Для описания эксперимента с несколькими возможными исходами используют такое вероятностное пространство \PSP:
	\begin{itemize}
		\item $\oO = \{1, 2, \ldots, n\}$
		\item $\Flg = 2^{\oO}$
		\item $\PP(\{i\}) = p_i$, где $\sum_{i=1}^n p_i = 1$
	\end{itemize}
\end{example}

\begin{example}[Повторение испытания с разновероятными исходами]
	Пусть теперь мы повторяем $k$ раз эксперимент, в котором возможно $n$ разновероятных исходов (ещё можно думать об этом, как о $k$-кратном выборе c возвращением из коробки с шарами $n$ цветов).
	\begin{itemize}
		\item $\oO = \{a_1a_2\ldots a_k |~a_i = 1, \ldots, n~~i < j \Rightarrow a_i \le a_j \}$
		\item $\Flg = 2^{\oO}$
		\item $\PP(\{a_1a_2\ldots a_k\}) = p_1^{t_1}p_2^{t_2}\ldots p_n^{t_n}$, где $\sum_{i=1}^n p_i = 1$, а $t_i = |a_1a_2\ldots a_k|_i$
	\end{itemize}
	Впоследствии мы определим на этом пространстве многомерную случайную величину «количество исходов каждого вида» («количество шаров каждого цвета»). У этой случайной величины будет распределение, которое мы назовём {\it мультиномиальным}.
\end{example}

Пространство из следующего примера яв-ся классическим вероятностным пространством, но помещено здесь, потому что перекликается с предыдущим примером.

\begin{example}[Выбор разноцветных шаров без возвращения]
	Пусть в коробке лежит $M$ шаров $n$ цветов. Пусть шаров цвета $i$ будет $p_i$ штук. Пусть мы $k$ раз вытаскиваем без возвращения шары из этой коробки.
	\begin{itemize}
		\item $\oO = \{a_1a_2\ldots a_k |~a_i = 1, \ldots, M,~~i \not = j \Rightarrow a_i \not = a_j \}$
		\item $\Flg = 2^{\oO}$
		\item $\PP(\{\omega\}) = \frac{1}{M^k}$
	\end{itemize}
	Впоследствии мы определим на этом пространстве многомерную случайную величину «количество шаров каждого цвета», которая будет иметь более сложную структуру, чем аналогичная величина из предыдущего примера. Её распределение носит название {\it многомерного гипергеометрического}, или просто {\it гипергеометрического}, в случае, когда всего 2 цвета.
\end{example}

\paragraph{Примеры бесконечных дискретных вероятностных пространств}

\begin{example}[Геометрическое распределение]
	Вероятностное пространство \PSP~ следующего вида
	\begin{itemize}
		\item $\oO = 0 \cup \mathbb N$
		\item $\Flg = 2^{\oO}$
		\item $\PP(k) = p q^k$, где $q = 1 - p$, $k \in 0 \cup \mathbb N$
	\end{itemize}
	описывает бесконечное повторение эксперимента до тех пор пока не случится успех. Элементарное событие $k$ соответствует получению первого успеха после $k$ неудачных попыток. Соответствующее распределение вероятностей называют {\it геометрическим распределением}.
\end{example}

\subsection{Геометрические вероятности}

\begin{definition}
Вероятностное пространство \PSP, где
\begin{itemize}
	\item $\oO \subset \RR^n$
	\item \Flg~--- имеющие объём (измеримые по Жордану) подмножества \oO
	\item $\PP(A)=\frac{|A|}{|\oO|}$, т. е. частному соответствующих объёмов
\end{itemize}
называется {\it геометрическим вероятностным пространством}.
\end{definition}

С помощью геометрической вероятности можно решать следующую задачу: пусть есть два студента. Пусть про каждого студента известно, что он приходит в столовую в случайное время в течение часа и обедает в течение 15 минут. Спрашивается вероятность встречи этих студентов. Решение заключается в том, чтобы отложить по координатным осям времена прихода студентов, отметить область точек, внутри которой студенты встречаются, и посчитать площадь этой области.

Говоря о геометрических вероятностях, можно упомянуть метод Монте-Карло (способ подсчёта чего-нибудь, (например, отношения площадей) с помощью многократного моделирования случайного процесса (например, бросания точки на фигуру)).
