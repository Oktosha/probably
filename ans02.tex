\section{Дискретные вероятностные пространства. Классическое определение вероятности. Примеры. Геометрические вероятности. Примеры.}

\subsection{Дискретные вероятностные пространства}

\paragraph{Определения}

\begin{definition}[Дискретное вероятностное пространство]
Вероятностное пространство \PSP~называется {\it дискретным вероятностным пространством}, если \oO~не более чем счётно.
\end{definition}

\begin{definition}[Классическое определение вероятности]
Вероятностное пространство \PSP~называется {\it классическим вероятностным пространством}, если:
	\begin{itemize}
	\item \oO~конечно, $|\oO| = n $
	\item $\Flg = 2^\oO$
	\item $\forall \omega \in \oO~~\PP(\omega) = \frac{1}{n}$
	\end{itemize}
\end{definition}


\paragraph{Примеры классических вероятностных пространств}

%дискретное равномерное

\begin{example}[Упорядоченный $k$-кратный выбор из $n$ объектов с возвращением]
Упорядоченный $k$-кратный выбор из $n$ объектов с возвращением описывается вероятностным пространством \PSP~следующего вида:
\begin{itemize}
		\item $\oO = $ все слова длины $k$ над алфавитом мощности $n$
		\item $\Flg = 2^{\oO}$
		\item $\PP(\{\omega\}) = \frac{1}{n^k}$, где $\omega \in \oO$
	\end{itemize}
\end{example}

%упорядоченный без возвращения
%неупорядоченный с возвращением
%неупорядоченный без возвращения

\paragraph{Примеры конечных (неклассических) дискретных вероятностных пространств}

\begin{example}[Распределение Бернулли]
	Вероятностное пространство \PSP~ следующего вида
	\begin{itemize}
		\item $\oO = \{0, 1\}$
		\item $\Flg = 2^{\oO}$
		\item $\PP(1) = p, \PP(0) = q$, где $q = 1 - p$
	\end{itemize}
	описывает некоторый однократный эксперимент, в котором $1$ соответствует успеху, $p$ --- вероятности успеха, а $0$ и $q$ --- провалу и его вероятности. Распределение вероятностей $\PP: \{\varnothing, \{0\}, \{1\}, \{0, 1\}\} \rightarrow \{0, q, p, 1\}$ называют {\it распределением Бернулли}.
\end{example}

\begin{example}[Схема Бернулли]
	Опыт, состоящий в $n$-кратном повторении некоторого эксперимента с вероятностью успеха $p$, и соответствующее ему вероятностное пространство
	\begin{itemize}
		\item $\oO = $ все последовательности нулей и единиц длины $n$.
		\item $\Flg = 2^{\oO}$
		\item $\PP(\omega) = p^k q^(n-k)$, где $q = 1 - p$, $k = |\omega|_1$
	\end{itemize}
	называют {\it схемой Бернулли}
\end{example}

%испытание с несколькими исходами
%мультиномиальная схема

%гипергеометрическая схема
%многомерная гипергеометрическая схема

\paragraph{Примеры бесконечных дискретных вероятностных пространств}

\begin{example}[Геометрическое распределение]
	Вероятностное пространство \PSP~ следующего вида
	\begin{itemize}
		\item $\oO = 0 \cup \mathbb N$
		\item $\Flg = 2^{\oO}$
		\item $\PP(k) = p q^k$, где $q = 1 - p$, $k \in 0 \cup \mathbb N$
	\end{itemize}
	описывает бесконечное повторение эксперимента до тех пор пока не случится успех. Элементарное событие $k$ соответствует получению первого успеха после $k$ неудачных попыток. Соответствующее распределение вероятностей называют {\it геометрическим распределением}.
\end{example}

\begin{example}[Распределение Пуассона]
%написать!
\end{example}

\subsection{Геометрические вероятности}

\begin{definition}
Вероятностное пространство \PSP, где
\begin{itemize}
	\item $\oO \subset \RR^n$
	\item \Flg~--- имеющие объём (измеримые по Жордану) подмножества \oO
	\item $\PP(A)=\frac{|A|}{|\oO|}$, т. е. частному соответствующих объёмов
\end{itemize}
называется {\it геометрическим вероятностным пространством}.
\end{definition}

С помощью геометрической вероятности можно решать следующую задачу: пусть есть два студента. Пусть про каждого студента известно, что он приходит в столовую в случайное время в течение часа и обедает в течение 15 минут. Спрашивается вероятность встречи этих студентов. Решение заключается в том, чтобы отложить по координатным осям времена прихода студентов, отметить область точек, внутри которой студенты встречаются, и посчитать площадь этой области.

Говоря о геометрических вероятностях, можно упомянуть метод Монте-Карло (способ подсчёта чего-нибудь, (например, отношения площадей) с помощью многократного моделирования случайного процесса (например, бросания точки на фигуру)).
