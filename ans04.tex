\section{Теорема о непрерывности в «нуле» вероятностной меры}

\begin{theorem}[О непрерывности в нуле вероятностной меры]
	Пусть $\langle \oO, \Alg \rangle$ --- измеримое пространство (\oO~---множество, \Alg~---алгебра, но не \gm алгебра). Пусть \PP~--- конечно-аддитивная вероятностная мера на $\langle \oO, \Alg \rangle$. Тогда следующие утверждения равносильны:
	\begin{itemize}
		\item \PP~счётно-аддитивная вероятностная мера
			$$\PP (\bigcup_{i=1}^{\infty}A_i) = \sum_{i=1}^{\infty}P(A_i)$$
			где $\{A_i\}$ --- последовательность попарно непересекающихся множеств, чьё объединение лежит в \Alg.
		\item \PP~непрерывна сверху
			$$\lim_{i\rightarrow\infty}\PP(A_i)=\PP(\bigcup_{i=1}^\infty A_i)$$
			где $\{A_i\}$ --- «неубывающая» последовательность вложенных множеств ($A_i \subset A_{i+1}$), чьё объединение лежит в \Alg.
		\item \PP~непрерывна снизу
			$$\lim_{i\rightarrow\infty}\PP(A_i)=\PP(\bigcap_{i=1}^\infty A_i)$$
			где $\{A_i\}$ --- «невозрастающая» последовательность вложенных множеств ($A_i \supset A_{i+1}$), чьё пересечение лежит в \Alg.
		\item \PP~непрерывна в нуле
			$$\lim_{i\rightarrow\infty}\PP(A_i)=0$$
			где $\{A_i\}$ --- «невозрастающая» последовательность вложенных множеств ($A_i \supset A_{i+1}$), чьё пересечение является пустым множеством.
	\end{itemize}
\end{theorem}

Эту теорему можно найти в Ширяеве с. 147. %сделать список литературы
Ниже конспект доказательства.
\begin{proof}
$1\Rightarrow 2$ Пусть $\{A_i\}$ --- «неубывающая» последовательность вложенных множеств ($A_i \subset A_{i+1}$), чьё объединение лежит в \Alg. Покажем, что

$$\lim_{i\rightarrow\infty}\PP(A_i)=\PP(\bigcup_{i=1}^\infty A_i)$$

Действительно, 

$$\bigcup_{i=1}^\infty A_i = A_1 \sqcup (A_2\setminus A_1) \sqcup (A_3\setminus A_3) \ldots$$

значит
\begin{multline}
	\PP(\bigcup_{i=1}^\infty A_i)=\\
	=\PP(A_1) + \sum_{i=1}^\infty \PP(A_{i+1}\setminus A_i)=\\
	=\PP(A_1) + \sum_{i=1}^\infty\left(\PP(A_{i+1}) - \PP(A_i)\right) = \\
	=\lim_{i\rightarrow\infty}\PP(A_i)
\end{multline}

$2 \Rightarrow 3$ 

Посмотрим на последовательность $\{A_1\setminus A_i\}$. Она удовлетворяет условиям из (2), а значит
$$\lim_{i\rightarrow\infty}\PP(A_1\setminus A_i) = \PP(\bigcup_{i=1}^\infty A_1\setminus A_i)$$
Тогда
\begin{multline}
\lim_{i\rightarrow\infty}\PP(A_i) =\\
= \PP(A_1) - \lim_{i\rightarrow\infty}\PP(A_1\setminus A_i) =\\
= \PP(A_1) - \PP(\bigcup_{i=1}^\infty A_1\setminus A_i) = \\
= \PP(A_1) - \PP(A_1\setminus \bigcap_{i=1}^\infty A_i) = \\
= \PP(\bigcap_{i=1}^\infty A_i)
\end{multline}

$3 \Rightarrow 4$. Кэп.

$4 \Rightarrow 1$. Пусть $\{A_i\}$ --- последовательность попарно непересекающихся событий, чьё объединение $A$ лежит в \Alg.

\begin{multline}
\sum_{i=1}^\infty\PP(A_i) = \lim_{n\rightarrow\infty}(\sum_{i = 1}^n\PP(A_i)) = \\
= \lim_{n\rightarrow\infty}(\PP(\bigcup_{i=1}^n A_i)) 
=  \lim_{n\rightarrow\infty}(\PP(\bigcup_{i = 1}^\infty A_i)-\PP(\bigcup_{i = n + 1}^\infty A_i)) =\\
= \PP(\bigcup_{i = 1}^\infty A_i) - \lim_{n\rightarrow\infty}\PP(\bigcup_{i = n + 1}^\infty A_i)) = \PP(\bigcup_{i = 1}^\infty A_i)
\end{multline} 

Поскольку последовательность $\{(\bigcup_{i = n + 1}^\infty A_i)_n\}$ является «невозрастающей» последовательностью множеств, чьё пересечение пусто, а значит $\lim_{n\rightarrow\infty}\PP(\bigcup_{i = n + 1}^\infty A_i)) = 0$.

\end{proof}