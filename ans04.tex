\section{Теорема о непрерывности в «нуле» вероятностной меры}

\begin{theorem}[О непрерывности в нуле вероятностной меры]
	Пусть $\langle \oO, \Alg \rangle$ --- измеримое пространство (\oO~---множество, \Alg~---алгебра, но не \gm алгебра). Пусть \PP~--- конечно-аддитивная вероятностная мера на $\langle \oO, \Alg \rangle$. Тогда следующие утверждения равносильны:
	\begin{itemize}
		\item \PP~счётно-аддитивная вероятностная мера
		\item \PP~непрерывна сверху
			$$lim_{i\rightarrow\infty}\PP(A_i)=\PP(\bigcup_{i=1}^\infty A_i$$
			где $\{A_i\}$ --- «неубывающая» последовательность вложенных множеств ($A_i \subset A_{i+1}$), чьё объединение лежит в \Alg.
		\item \PP~непрерывна снизу
			$$lim_{i\rightarrow\infty}\PP(A_i)=\PP(\bigcap_{i=1}^\infty A_i$$
			где $\{A_i\}$ --- «невозрастающая» последовательность вложенных множеств ($A_i \supset A_{i+1}$), чьё пересечение лежит в \Alg.
		\item \PP~непрерывна в нуле
			$$lim_{i\rightarrow\infty}\PP(A_i)=0$$
			где $\{A_i\}$ --- «невозрастающая» последовательность вложенных множеств ($A_i \supset A_{i+1}$), чьё пересечение является пустым множеством.
	\end{itemize}
\end{theorem}