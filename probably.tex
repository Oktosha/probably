\documentclass[russian, twoside, 12pt, a5paper]{article}
\usepackage[a5paper]{geometry}

\sloppy %грубая борьба с переполнениями

%usual magic
\usepackage{cmap}
\usepackage[utf8]{inputenc}
\usepackage[T2A]{fontenc}
\frenchspacing

\usepackage{xcolor}
\usepackage{hyperref}
\usepackage{graphicx}
\definecolor{linkcolor}{HTML}{799B03} % цвет ссылок
\definecolor{urlcolor}{HTML}{799B03} % цвет гиперссылок
\hypersetup{linkcolor=linkcolor,urlcolor=urlcolor,colorlinks=true}

%многострочные формулы и другое
\usepackage{amsmath}

%шрифты
\usepackage{amssymb}

%стили теорем и определений
\usepackage{amsthm}
\theoremstyle{theorem}
\newtheorem{statement}{Утверждение}
\newtheorem{theorem}{Теорема}

\theoremstyle{definition}
\newtheorem{definition}{Определение}
\newtheorem{example}{Пример}

%en to ru
\renewcommand{\contentsname}{Содержание}
\renewcommand{\proofname}{Доказательство}

%команды для ускорения письма
\newcommand{\PP}{\ensuremath{\mathbf{P}}}
\newcommand{\Alg}{\ensuremath{\mathcal{A}}}
\newcommand{\Flg}{\ensuremath{\mathcal{F}}}
\newcommand{\RR}{\ensuremath{\mathbb{R}}}
\newcommand{\oO}{\ensuremath{\Omega}}
\newcommand{\PSP}{\ensuremath{\langle \Omega, \mathcal F, \mathbf P \rangle}}
\newcommand{\gm}{\ensuremath{\sigma}-}

\begin{document}

\title{Ответы на вопросы по курсу «Теория Вероятностей»
	\thanks{Лектор: Жуковский Максим Евгеньевич\endgraf
	       	Место: ФИВТ МФТИ}
}

\author{Колодзей Дарья, 394
	\thanks{
		Спасибо Алексею Журавлёву за конспекты и билеты\endgraf
		Спасибо Павлу Ахтямову за конспекты\endgraf
		Спасибо Дмитрию Иващенко за печатные конспекты
		}}

\date{осенний семестр 2014}

\maketitle
\tableofcontents

\section{Вероятностное пространство ($\Omega$, $\EuScript F$, $P$). Аксиомы Колмогорова.}
\section{Дискретные вероятностые пространства. Классическое определение вероятности. Примеры. Геометрические вероятности. Примеры.}

\subsection{Дискретные вероятностные пространства}

\paragraph{Определения}

\begin{definition}[Дискретное вероятностное пространство]
Вероятностное пространство \PSP~называется {\it дискретным вероятностным пространством}, если \oO~не более чем счётно.
\end{definition}

\begin{definition}[Классическое определение вероятности]
Вероятностное пространство \PSP~называется {\it классическим вероятностным пространством}, если:
	\begin{itemize}
	\item \oO~конечно, $|\oO| = n $
	\item $\Flg = 2^\oO$
	\item $\forall \omega \in \oO~~\PP(\omega) = \frac{1}{n}$
	\end{itemize}
\end{definition}


\paragraph{Примеры классических вероятностных пространств}

%дискретное равномерное

\begin{example}[Упорядоченный $k$-кратный выбор из $n$ объектов с возвращением]
Упорядоченный $k$-кратный выбор из $n$ объектов с возвращением описывается вероятностным пространством \PSP~следующего вида:
\begin{itemize}
		\item $\oO = $ все слова длины $k$ над алфавитом мощности $n$
		\item $\Flg = 2^{\oO}$
		\item $\PP(\{\omega\}) = \frac{1}{n^k}$, где $\omega \in \oO$
	\end{itemize}
\end{example}

%упорядоченный без возвращения
%неупорядоченный с возвращением
%неупорядоченный без возвращения

\paragraph{Примеры конечных (неклассических) дискретных вероятностных пространств}

\begin{example}[Распределение Бернулли]
	Вероятностное пространство \PSP~ следующего вида
	\begin{itemize}
		\item $\oO = \{0, 1\}$
		\item $\Flg = 2^{\oO}$
		\item $\PP(1) = p, \PP(0) = q$, где $q = 1 - p$
	\end{itemize}
	описывает некоторый однократный эксперимент, в котором $1$ соответствует успеху, $p$ --- вероятности успеха, а $0$ и $q$ --- провалу и его вероятности. Распределение вероятностей $\PP: \{\varnothing, \{0\}, \{1\}, \{0, 1\}\} \rightarrow \{0, q, p, 1\}$ называют {\it распределением Бернулли}.
\end{example}

\begin{example}[Схема Бернулли]
	Опыт, состоящий в $n$-кратном повторении некоторого эксперимента с вероятностью успеха $p$, и соответствующее ему вероятностное пространство
	\begin{itemize}
		\item $\oO = $ все последовательности нулей и единиц длины $n$.
		\item $\Flg = 2^{\oO}$
		\item $\PP(\omega) = p^k q^(n-k)$, где $q = 1 - p$, $k = |\omega|_1$
	\end{itemize}
	называют {\it схемой Бернулли}
\end{example}

%испытание с несколькими исходами
%мультиномиальная схема

%гипергеометрическая схема
%многомерная гипергеометрическая схема

\paragraph{Примеры бесконечных дискретных вероятностных пространств}

\begin{example}[Геометрическое распределение]
	Вероятностное пространство \PSP~ следующего вида
	\begin{itemize}
		\item $\oO = 0 \cup \mathbb N$
		\item $\Flg = 2^{\oO}$
		\item $\PP(k) = p q^k$, где $q = 1 - p$, $k \in 0 \cup \mathbb N$
	\end{itemize}
	описывает бесконечное повторение эксперимента до тех пор пока не случится успех. Элементарное событие $k$ соответствует получению первого успеха после $k$ неудачных попыток. Соответствующее распределение вероятностей называют {\it геометрическим распределением}.
\end{example}

\begin{example}[Распределение Пуассона]
%написать!
\end{example}

\subsection{Геометрические вероятности}

\begin{definition}
Вероятностное пространство \PSP, где
\begin{itemize}
	\item $\oO \subset \RR^n$
	\item \Flg~--- имеющие объём (измеримые по Жордану) подмножества \oO
	\item $\PP(A)=\frac{|A|}{|\oO|}$, т. е. частному соотвествующих объёмов
\end{itemize}
называется {\it геометрическим вероятностным пространством}.
\end{definition}

С помощью геометрической вероятности можно решать следующую задачу: пусть есть два студента. Пусть про каждого студента известно, что он приходит в столовую в случайное время в течение часа и обедает в течение 15 минут. Спрашивается вероятность встречи этих студентов. Решение заключается в том, чтобы отложить по координатным осям времена прихода студентов, отметить область точек, внутри которой студенты встречаются, и посчитать площадь этой области.

Говоря о геометрических вероятностях, можно упомянуть метод Монте-Карло (способ подсчёта чего-нибудь, (например, отношения площадей) с помощью многократного моделирования случайного процесса (например, бросания точки на фигуру)).
\section{Условные вероятности. Формула полной вероятности. Формула Байеса. Примеры}

\begin{definition}[Условная вероятность]
	Пусть \PSP~---вероятностное пространство, $A, B \in \Flg$. Тогда {\it условной вероятностью} события $A$ при условии, что произошло событие $B$, называют величину
	$$\PP(A|B)=\frac{\PP(A\cap B)}{\PP(B)}, \mbox{если $\PP(B) > 0$}$$
	$$\PP(A|B) = 0, \mbox{если $\PP(B) = 0$}$$
\end{definition}

\begin{statement}
Пусть \PSP~--- вероятностное пространство. Пусть $B\in\Flg,~\PP(B) > 0$. Тогда $\PP(\cdot|B): \Flg \rightarrow \RR$ является счётно-аддитивной вероятностной мерой над измеримым пространством $\langle \oO, \Flg \rangle$.
\end{statement}

\begin{proof}
	Проверим, что для $\PP(\cdot|B)$ выполняется определение вероятностной меры. Действительно:
	$$\forall A \in \Flg~~\PP(A|B)=\frac{\PP(A\cap B)}{\PP(B)}\ge 0$$
	$$\PP(\oO|B)=\frac{\PP(\oO\cap B)}{\PP(B)} = 1$$
	Пусть $\{A_i\}$ --- последовательность попарно непересекающихся событий. Поскольку $\Flg$ --- \gmалгебра, то их объединение $A$ тоже принадлежит $\Flg$. Проверим, что
	$$\PP(A|B)=\sum_{i=1}^\infty\PP(A_i|B)$$
	Действительно,
	\begin{multline}
		\sum_{i=1}^\infty\PP(A_i|B)
		 =\sum_{i=1}^\infty\frac{\PP(A_i \cap B)}{\PP(B)} = \\
		 =\frac{1}{\PP(B)}\sum_{i=1}^\infty\PP(A_i \cap B)
		 =\frac{1}{\PP(B)}\PP(A\cap B) = \\
		 =\PP(A|B)
	\end{multline}
\end{proof}

\begin{theorem}[Формула полной вероятности]
Пусть \PSP~--- вероятностное пространство. Пусть $\oO=\bigsqcup_{i=1}^\infty B_i$, пусть $A, B_1, B_2, \ldots \in \Flg$. Тогда
$$\PP(A)=\sum_{i=1}^{\infty}\PP(A|B_i)\PP(B_i)$$
\end{theorem}

\begin{proof}
Заметим, что $\PP(A\cap B_i) = \PP(A|B_i)\PP(B_i)$ является верным равенством и в случае, когда $\PP(B_i) = 0$, и в случае, когда $\PP(B_i) > 0$. А значит
$$\sum_{i=1}^{\infty}\PP(A|B_i)\PP(B_i)=
\sum_{i=1}^{\infty}\PP(A\cap B_i)=
\PP(A)$$
\end{proof}
Заметим, что в конечных случаях формула полной вероятности тоже работает.

\begin{theorem}[Формула Байеса]
Пусть \PSP~--- вероятностное пространство. Пусть $\oO=\bigsqcup_{i=1}^\infty B_i$, пусть $A, B_1, B_2, \ldots \in \Flg$. Пусть также $\PP(A) > 0$ Тогда
$$\PP(B_k|A)=\frac{\PP(A|B_k)\PP(B_k)}{\sum_{i=1}^\infty\PP(A|B_i)\PP(B_i)}$$
\end{theorem}

\begin{proof}
$$\PP(A\cap B_k) = \PP(A|B_k) \PP(B_k)$$
$$\PP(A\cap B_k) = \PP(B_k|A) \PP(A)$$
$$\PP(B_k|A) \PP(A) = \PP(A|B_k) \PP(B_k)$$
$$\PP(B_k|A) = \frac{\PP(A|B_k) \PP(B_k)}{\PP(A)}$$
осталось расписать $\PP(A)$ по формуле полной вероятности и получить желаемое.
\end{proof}

Смысл формулы Байеса можно понимать так: $B_i$ --- это гипотезы, а $A$ --- результат эксперимента. Нам известна {\it априорная} вероятность $\PP(A|B_i)$ получения результата $A$ при выполнении гипотезы $B_i$. Теперь, зная результат эксперимента, мы хотим узнать {\it апостериорную} вероятность того, что гипотеза $B_k$ верна.

Например, с помощью формулы Байеса можно решать какую-нибудь задачу про смерть лорда Вайла, которого хотят отравить или зарезать дворецкий, сын и жена.

Приведём ещё пример решения задачи про шары помощью формулы Байеса.
\begin{example}
	Пусть в ящике $n$ шаров, из них $k$ белых. Шары извлекаются без возвращений равновероятно. Какова вероятность на $j$-ом шаге вытащить белый шар?

	Итак, вероятностное пространство, соотвествующее $j$-кратному выбору без возвращения таково:

		\begin{itemize}
		\item $\oO = \{a_1a_2 \ldots a_j |~a_i = 1, 2, \ldots, n\}$
		\item $\Flg = 2^\oO$
		\item $P(\{a_1a_2 \ldots a_j\}) = \frac{1}{\frac{(n)!}{(n-j)!}}$
		\end{itemize}

	Доказывать будет индукцией по числу шаров и шагов. Обозначим за $A_{j,n,k}$ событие «вытащить белый шар на $j$-ом шаге, если в урне изначально было $n$ шаров, из которых $k$ белых» и за $B_{j,n,k}$ событие «вытащить чёрный шар на $j$-ом шаге, если в урне изначально было $n$ шаров, из которых $k$ белых».

	{\it База.} $A_{1,n,k}=\frac{k}{n}$.
	
	{\it Доказательство перехода.} 
	\begin{multline}
	\PP(A_{j,n,k}) = \\
	 = \PP(A_{j,n,k}|A_{1,n,k})\PP(A_{1,n,k}) + \PP(A_{j,n,k}|B_{1,n,k})\PP(B_{1,n,k}) = \\ 
	 = \PP(A_{j-1,n-1,k-1})\PP(A_{1,n,k}) + \PP(A_{j-1,n-1,k})\PP(B_{1,n,k})
	 \end{multline}
\end{example}
\section{Теорема о непрерывности в «нуле» вероятностной меры}

\begin{theorem}[О непрерывности в нуле вероятностной меры]
	Пусть $\langle \oO, \Alg \rangle$ --- измеримое пространство (\oO~---множество, \Alg~---алгебра, но не \gm алгебра). Пусть \PP~--- конечно-аддитивная вероятностная мера на $\langle \oO, \Alg \rangle$. Тогда следующие утверждения равносильны:
	\begin{itemize}
		\item \PP~счётно-аддитивная вероятностная мера
		\item \PP~непрерывна сверху
			$$lim_{i\rightarrow\infty}\PP(A_i)=\PP(\bigcup_{i=1}^\infty A_i$$
			где $\{A_i\}$ --- «неубывающая» последовательность вложенных множеств ($A_i \subset A_{i+1}$), чьё объединение лежит в \Alg.
		\item \PP~непрерывна снизу
			$$lim_{i\rightarrow\infty}\PP(A_i)=\PP(\bigcap_{i=1}^\infty A_i$$
			где $\{A_i\}$ --- «невозрастающая» последовательность вложенных множеств ($A_i \supset A_{i+1}$), чьё пересечение лежит в \Alg.
		\item \PP~непрерывна в нуле
			$$lim_{i\rightarrow\infty}\PP(A_i)=0$$
			где $\{A_i\}$ --- «невозрастающая» последовательность вложенных множеств ($A_i \supset A_{i+1}$), чьё пересечение является пустым множеством.
	\end{itemize}
\end{theorem}

\end{document}
